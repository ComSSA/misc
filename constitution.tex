% Please adhere to having exactly one clause or sentence per line of source.
% At the expense of 80-column purity, this makes diff output much more useful.

\documentclass[a4paper,12pt]{article}
\usepackage[utf8]{inputenc}
\usepackage[T1]{fontenc}
\usepackage{parskip}
\usepackage{hyperref}
\usepackage{graphicx}
\usepackage{enumerate}

\hypersetup{
	pdfauthor=ComSSA,
	pdftitle=The Constitution of the Computer Science Students Association
}

\title{\scshape
	The Constitution\\
	of the\\
	Computer Science Students Association
}

\date{October 15, 2014}

\author{} % suppress 'LaTeX Warning: No \author given.'

\pagenumbering{gobble}
\thispdfpagelabel0

\raggedright

\begin{document}

\maketitle

\vspace{1in}

\begin{center}
	\includegraphics[width=3in]{logo/delan/comssalogo_crop_black.eps}
\end{center}

\newpage

\pagenumbering{arabic}

\section{Preliminary}

\subsection{Name}

The Club shall be known as the \textit{Computer Science Students Association}, also by the short name \textit{ComSSA} and hereafter referred to as ComSSA.

\subsection{Objective}

The objective of this Constitution is to:

\begin{enumerate}[a)]
	\item Provide a framework for the operation of the Club, and
	\item Assist the Members who operate the Club in a way that is consistent with the ethos of the Club.
\end{enumerate}

\subsection{Definition and Interpretation}

\textbf{``ComSSA''} is the short name for the Computer Science Students Association.

\textbf{``Club''} refers to ComSSA.

\textbf{``Constitution''} refers to this document.

\textbf{``AGM''} stands for ``Annual General Meeting''.

\textbf{``SGM''} stands for ``Special General Meeting''.

\textbf{``Club Year''} refers to the period of time between the AGM of one year until the AGM of the next year.

\textbf{``Member''} refers to any member of ComSSA, unless otherwise qualified.

\textbf{``Committee''} refers to the group of Members who are responsible for the operation of the Club.

\textbf{``Executive Committee''} refers to the subset of Committee Members who are also office bearers.

\textbf{``OCM''} stands for ``Ordinary Committee Member''.

\textbf{``Club Asset''} refers to any item belonging to ComSSA, be it monetary, physical, abstract or otherwise.

\textbf{``Majority''} is defined as the nearest whole number to half of a cardinality, that is strictly greater than half of that cardinality.

\subsection{Registration}

As ComSSA is a part of Curtin University, it must officially register as a Curtin Student Guild-affiliated club each year.

\subsection{Limit of the ComSSA Constitution}

This Constitution is restricted and subject to the Curtin University Statutes and the Curtin Student Guild Regulations.

\section{Objectives of the Club}

\subsection{Objectives}

The objectives of the Club are:

\begin{enumerate}[a)]
	\item To provide social events for students and staff of the Department of Computing,
	\item To provide support for students entering university life,
	\item To provide support for students who need representation with the Department of Computing, and
	\item To liaise with the Department of Computing and function as a forum for student and staff relations.
\end{enumerate}

\section{Membership}

\subsection{Limits of Membership}

Membership of the Club is unrestricted excepting a unanimous vote by the Committee.

\subsection{Membership Length}

All memberships shall remain active from full payment of the membership fee until the beginning of the first day of Teaching Week 1 of Semester 1 of the following academic year, as set forth in the Curtin University academic calendar for that year.

\section{Fees and Membership Costs}

\subsection{Membership Fee}

The membership fee for students with a full, financial Curtin Student Guild membership shall be five (5) Australian Dollars, or a Curtin Student Guild club registration voucher.

Registration for all other Members shall be ten (10) Australian dollars.

\subsection{Compulsory Fees}

There shall be no compulsory fees charged to Members apart from the membership fee.

\subsection{Costs for Members to Attend ComSSA Events}

The maximum cost for Members to attend any official ComSSA event shall be ten (10) Australian dollars or 20\% above cost price, whichever is greater.

\section{Committee}

\subsection{Committee Structure}

The Committee shall consist of the following voting office bearers:

\begin{enumerate}[a)]
	\item President,
	\item Vice-President,
	\item Treasurer, and
	\item Secretary.
\end{enumerate}

The Committee shall consist of the following voting non-office bearers:

\begin{enumerate}[a)]
	\setcounter{enumi}{4}
	\item Three (3), five (5) or seven (7) Ordinary Committee Members.
\end{enumerate}

\subsection{Length of Term}

Positions shall be held for one year; from the last day of the second semester (including the exam weeks) to the same time in the following year, as defined by the Curtin University acacdemic calendar.

\subsection{Committee Eligibility}

To be considered for any committee position, a Nominee must meet all of the following criteria:

\begin{enumerate}[a)]
	\item The Nominee must be a current, full time student at Curtin University, studying at the Bentley campus, and
	\item The Nominee must be a ComSSA member
\end{enumerate}

\subsection{Executive Committee Eligibility}

To be considered for an office bearer position, a Nominee must meet all of the following criteria:

\begin{enumerate}[a)]
	\item The Nominee must be a Curtin Student Guild member, and
	\item At least one of the following must be true for the Nominee:
	\begin{enumerate}[i)]
		\item The Nominee must be studying at least one course which has greater than 50\% of its credit points assigned to units run by the Department of Computing, and/or
		\item The Nominee must be studying at least one course which will yield a degree which is deemed as ``relevant to computing'' by an academic staff member of the Department of Computing in the position of Senior Lecturer or higher.
	\end{enumerate}
\end{enumerate}

\subsection{Ordinary Committee Members}

The method of selection for Ordinary Committee Member positions shall be determined by the Committee, and shall be limited to either:

\begin{enumerate}[a)]
	\item A Majority vote by Members attending an AGM or SGM, or
	\item Appointment by the Committee.
\end{enumerate}

\subsection{Quorum}

Quorum is satisfied when the following conditions are met:

\begin{enumerate}[a)]
	\item Three (3) or more Executive Members must be present, and
	\item Five (5) total Committee Members must be present.
\end{enumerate}

A Committee Member may qualify to be counted in both of these conditions.

\subsection{Vacancy of Position}

Any position will become vacant upon any of the following criteria:

\begin{enumerate}[a)]
	\item The current occupant's written notice to the Committee,
	\item Absence from two (2) consecutive meetings without given notice, upon verification of the Committee,
	\item Absence from four (4) consecutive meetings, upon verification of the Committee, or
	\item No longer being a Member of ComSSA.
\end{enumerate}

\subsection{Vacancy or Incapacitation}

\begin{enumerate}[a)]
	\item If any Executive Committee Member vacates their position or becomes incapacitated, then the President shall acquire all of their powers and responsibilities until an SGM is called and the position is filled.
	\item The President may appoint an interim from the membership to temporarily fill the position until an SGM is called with the approval of the Committee.
	\item An SGM must be called within four (4) weeks of vacancy of a position.
\end{enumerate}

\section{Powers and Responsibilities}

\subsection{Duties of the Committee}

The Committee has the power and responsibility to:

\begin{enumerate}[a)]
	\item Plan activities in accordance with the Club objectives and inform all Members of these activities,
	\item Formulate Policy in accordance with the Club objectives,
	\item Strive to consistently have available committee members on campus as often as possible,
	\item Act according to all enacted Policy,
	\item Raise and spend funds in accordance with the current Spending Policy,
	\item Act in the best interests of the Club, and
	\item Operate the Club in an ethical manner.
\end{enumerate}

\subsection{Duties of the President}

The President has the power and responsibility to:

\begin{enumerate}[a)]
	\item Represent the Club in any matters relating to ComSSA,
	\item Co-ordinate and supervise the Committee,
	\item Familiarise the Members with the objectives of the club, and
	\item Call for Special General Meetings.
\end{enumerate}

\subsection{Duties of the Vice-President}

The Vice-President has the power and responsibility to:

\begin{enumerate}[a)]
	\item Assist the President in carrying out their responsibilities.
\end{enumerate}

\subsection{Duties of the Treasurer}

The Treasurer has the power and responsibility to:

\begin{enumerate}[a)]
	\item Keep a record of Club property and finances,
	\item Prepare a report of income and expenditure during the Club Year to be provided during an AGM or SGM,
	\item Write or make necessary amendments to the Spending Policy,
	\item Ensure the Committee's adherence to the Spending Policy,
	\item Familiarise and report to the Committee the financial status of the Club,
	\item Suspend a Committee Member upon a violation of the Spending Policy.
\end{enumerate}

\subsection{Duties of the Secretary}

The Secretary has the power and responsibility to:

\begin{enumerate}[a)]
	\item Record and keep minutes of all Committee meetings, and to ensure that those minutes are signed by the chairperson of the meeting,
	\item Manage correspondence of the Club,
	\item Manage a list of current Members,
	\item Keep, publish and provide to any Member the Constitution, Policies and meeting minutes of the Club, and
\end{enumerate}

\subsection{Duties of Ordinary Committee Members}

The OCMs have the power and responsibility to:

\begin{enumerate}[a)]
	\item Vote as Committee Members,
	\item Perform tasks as delegated by the Committee,
	\item Assist the Committee in their duties, and
	\item Ensure the proper conduct of the Executive Committee Members.
\end{enumerate}

\subsection{Duties of the Members}

Members have the power and responsibility to:

\begin{enumerate}[a)]
	\item Vote during an AGM or SGM, and
	\item Become a part of the Committee.
\end{enumerate}

\section{Suspension and Removal of Committee Members}

\subsection{Suspension of Ordinary Committee Members}

\begin{enumerate}[a)]
	\item An Ordinary Committee Member may be placed under a state of suspension by two Executive Committee Members.
	\item Once suspended, the OCM loses all powers and responsibilities associated with their position, until one of the following conditions is met:
	\begin{enumerate}[i)]
		\item A period of fourteen (14) days passes since the suspension, or
		\item A Committee Meeting is held, during which the removal of the OCM in question is voted upon.
	\end{enumerate}
\end{enumerate}

\subsection{Removal of Ordinary Committee Members}

\begin{enumerate}[a)]
	\item Ordinary Committee Members can be removed from their position via an Absolute Majority vote by the Executive Committee.
	\item The OCM in question may be removed from the meeting during discussion. This will be decided upon by a Simple Majority vote of present Committee Members excluding said OCM. 
	\item Any two Committee Members may request that the OCM in question be removed from the meeting during the vote.
	\item If the OCM is removed:
	\begin{enumerate}[i)]
		\item The Committee must supply to the former OCM justification for removal.
	\end{enumerate}
\end{enumerate}

\subsection{Removal of Executive Members}

\subsubsection{Requests for Removal}

Any Member may request the removal of an Executive Committee Member from office. The request must be given in writing, and will only be considered if:

\begin{enumerate}[a)]
	\item The request includes reasons for removal from office, and
	\item The request includes the signatures of at least ten (10) Members or 10\% of the current membership, whichever is greater.
\end{enumerate}

\subsubsection{Consideration of Request}

A Committee Meeting must be held within fourteen (14) days of receipt of a request.

\begin{enumerate}[a)]
	\item The Members that requested removal must be notified of the time and place of the Committee Meeting.
	\item The Member(s) may present their reasons in person to the Committee.
	\item The Executive Committee Member in question may explain their actions to the Committee and the Member(s) in attendance.
	\item The Executive Committee Member will be removed from their position upon an Absolute Majority vote by the Committee, with the exception of the Executive Committee Member being considered for removal.
	\begin{enumerate}[i)]
		\item Any two Committee Members can request that the Executive Committee Member in question be removed from the meeting during the vote.
		\item Any one Committee Member can request that the Executive Committee Member in question be removed from the meeting during discussion.
	\end{enumerate}
	\item The Committee must supply to the Executive Committee Member in question, and the raising Member(s), justification for the decision made.
\end{enumerate}

\subsubsection{Removal of Executive Members}

An Executive Committee Member can be removed from office in any of the following ways:

\begin{enumerate}[a)]
	\item Granting of request for removal,
	\item A unanimous vote by the Committee with the exception of the Executive Committee Member being considered for removal:
	\begin{enumerate}[i)]
		\item Any one Committee Member may request that the Executive Committee Member in question be removed from the meeting during discussion, but the Executive Committee Member must be given a chance to explain their actions.
		\item Any two Committee Members can request that the Executive Committee Member in question be removed from the meeting during the vote.
	\end{enumerate}
\end{enumerate}

Upon removal from the Committee:

\begin{enumerate}[a)]
	\item The former Executive Committee Member must yield all Club Assets in their possession to the remaining Executive Committee Members.
	\item The Committee must supply to the former Executive Committee Member justification for removal.
	\item An SGM must be called within four (4) weeks of their removal.
\end{enumerate}

\section{Dismissal of Committee}

\subsection{Request for Dismissal}

Any Member may request for the dismissal of the entire Committee. The request must be given in writing, and will only be considered if:

\begin{enumerate}[a)]
	\item The request includes reasons for removal from office, and
	\item The request includes the signatures of at least twenty (20) members or 20\% of the current membership, whichever is greater.
\end{enumerate}

\subsection{Receipt of Request}

An SGM must be called within fourteen (14) days of receipt of request. The agenda shall include:

\begin{enumerate}[a)]
	\item Justification for dismissal of the Committee by the Member(s) making the request,
	\item The Committee explanation for the actions detailed in the justification, and
	\item A vote by the attending Members for dismissal of the entire Committee.
	\begin{enumerate}[i)]
		\item The method of voting for dismissal of the Committee shall be a secret ballot supervised by an independent source.
	\end{enumerate}
	\item Election of the new Committee (if necessary).
\end{enumerate}

\subsection{Confirmation of Dismissal}

Upon dismissal of the Committee:

\begin{enumerate}[a)]
	\item The dismissed Committee must yield all Club Assets in their possession to the newly elected Committee.
\end{enumerate}

\subsection{Right to Appeal}

The dismissed Committee may stand for re-election at the SGM, and as such has no right to appeal.

\section{Annual General Meetings}

\subsection{Time to be Held}

The AGM is to be held in the last four teaching weeks of Semester Two, as defined by the Curtin University academic calendar.

\subsection{Notice to be Given}

At least fourteen (14) days of notice is to be given to all Members detailing date, time and location of an AGM.

\subsection{Agenda}

The agenda shall consist of, in order:

\begin{enumerate}[a)]
	\item A report from each Executive Committee Member,
	\item Ratification of the Constitution,
	\item Election of the Executive Committee, and
	\item Any matter that is in writing, signed by at least ten (10) Members or 10\% of the current membership, whichever is greater.
\end{enumerate}

\subsection{Election of Positions}

The method of election shall be a secret ballot supervised by an independent source.

\subsection{Quorum}

Quorum for the AGM shall be twenty (20) Members or a Majority of the listed Members, whichever is fewer.

\subsection{Voting}

\begin{enumerate}[a)]
	\item The method of voting, with the exception of election of positions, shall be decided by the President, or in their absence, the Vice-President.
	\item Any matter that is voted upon shall be deemed to be passed if a Majority of attending votes are in favour.
\end{enumerate}

\section{Special General Meetings}

\subsection{Convening}

An SGM can be called for by any of the following means:

\begin{enumerate}[a)]
	\item Written notice by a Majority of the Committee,
	\item Written notice from the President, or
	\item Written notice by at least ten (10) Members or 10\% of the current membership, whichever is greater.
\end{enumerate}

\subsection{Notice to be Given}

At least seven (7) days but no more than four (4) weeks of notice is to be given to all Members detailing date, time and location of an SGM.

\subsection{Agenda}

The agenda shall consist of, in order:

\begin{enumerate}[a)]
	\item A report from each Executive Committee Member,
	\item Election of any vacant office bearer positions,
	\item Any matter in writing signed by at least ten (10) Members or 10\% of the current membership, whichever is greater, and
	\item Any matter in writing signed by a Majority of the Committee.
\end{enumerate}

\subsection{Quorum}

Quorum for an SGM shall be twenty (20) Members or a Majority of the current membership, whichever is fewer.

\subsection{Election of Positions}

The method of election shall be a secret ballot supervised by an independent source.

\subsection{Voting}

\begin{enumerate}[a)]
	\item The method of voting, with the exception of election of positions and dismissal of the Committee, shall be decided by the President, or in their absence, the Vice-President.
	\item Any matter that is voted upon shall be deemed to be passed if a Majority of attending votes are in favour.
\end{enumerate}

\section{Policies}

\subsection{Definition}

A Policy is a formal, standardised document, used to record a rule set or process related to the running of the Club.

\subsection{Forming Policy}

Except where otherwise stated in this document, any ComSSA Member (or group of Members, including any and all Committee Members) may form a Policy to be considered by the Committee.

\subsection{Consideration Conditions}

In order for a Policy to be considered by the Committee, it must meet the Consideration Conditions:

\begin{enumerate}[a)]
	\item Be in writing,
	\item State the full name of the person(s) that contributed to the Policy, hereby referred to as the Submitter(s),
	\item Be endorsed by one (1) Committee Member; this may be one of the Submitter(s), and
	\item State the full name of the Committee Member that endorsed the Policy, hereby referred to as the Endorser.
\end{enumerate}

\subsection{Considering and Enacting a Policy}

When a Policy meets the Consideration Conditions, it may be considered by the Committee.

\begin{enumerate}[a)]
	\item In order for a Policy to be voted upon in a meeting, it must be included in the agenda for said meeting.
	\begin{enumerate}[i)]
		\item The Policy document must be included along with the agenda upon circulation to Committee Members. The intent of this condition is to allow Committee Members to process a Policy and its possible ramifications before voting.
	\end{enumerate}
	\item At the Committee meeting (hereafter the Policy Meeting) the Policy will undergo an initial out-loud reading in its entirety by the chairperson.
	\item The Policy will be discussed by the Committee Members present at the Policy Meeting.
	\item Before voting takes place, amendments may be made to the Policy by the Submitter(s).
	\begin{enumerate}[i)]
		\item If any amendments are made to a Policy, they must be approved by all Submitter(s).
		\item At any point during the Policy Meeting, any of the Submitter(s) may remove their name(s) from the Policy.
		\item At any point during the Policy Meeting, and with the permission of all other Submitter(s) (or with nobody's permission if the Policy is has been left without any Submitters), someone may add themselves to the list of Submitter(s) and make amendments.
		\item If the Policy is without Submitter(s), with nobody willing to become a Submitter, discussion is dropped.
		\item At any time during discussion, the Endorser may cease endorsing the Policy. If no Committee Members endorse the Policy, discussion is dropped.
	\end{enumerate}
	\item Once all Submitter(s) and the Endorser are satisfied with the state of the Policy, the Policy will go to a vote by the Committee Members present at the Policy Meeting. If a Committee member is not present at the Policy Meeting, they may choose to vote remotely as an absentee.
	\begin{enumerate}[i)]
		\item Any two (2) Committee Members attending the Policy Meeting may request to have voting postponed until the next Committee meeting (which becomes the Policy Meeting upon reenactment of this process).
		\item If said Committee member(s) are not present at the next Policy Meeting, they forfeit their right to vote on the Policy.
		\item A Policy Meeting may not be postponed more than once per Policy.
	\end{enumerate}
	\item Policy is considered Enacted when it receives a Majority vote of approval by the Committee Members attending the Policy Meeting.
\end{enumerate}

\subsection{Nullification of a Policy}

A Policy can be nullified by the Committee via a Majority vote at a Committee meeting.

\subsection{Amendment Consideration Conditions}

A Policy may be amended after enactment.

In order for a Policy amendment to be considered by the Committee, it must meet the Consideration Conditions:

\begin{enumerate}[a)]
	\item Be in writing,
	\item Be based on an existing enacted Policy,
	\item State the full name of the person(s) that contributed to the Policy amendment, hereby referred to as the Amender(s),
	\item Be endorsed by one (1) Committee member; this may be one of the Amender(s), and
	\item State the full name of the Committee member that endorsed the Policy, hereby referred to as the Endorser.
\end{enumerate}

\subsection{Amendment of a Policy}

The process for amending a Policy is as follows:

For the duration of the following procedure, Amender(s) refers to the Amender(s) for the current amendment only.

\begin{enumerate}[1)]
	\item In order for a Policy amendment to be voted upon in a meeting, it must be included in the agenda for said meeting.
	\begin{enumerate}[i)]
		\item The amended Policy document must be included along with the agenda upon circulation to Committee Members.
		\item Inherently, a Policy amendment may not be voted upon with less than 48 hours notice to all Committee Members.
	\end{enumerate}
	\item At the next Committee meeting (hereafter the Policy Meeting) the amended Policy will undergo an initial out-loud reading in its entirety by the chairperson.
	\item The amended Policy will be discussed by the Committee Members present at the Policy Meeting.
	\item Before voting takes place, further amendments may be made to the Policy by the Amender(s).
	\begin{enumerate}[i)]
		\item If any further amendments are made to a Policy, they must be approved by all Amender(s).
		\item At any point during the Policy Meeting, any of the Amender(s) may remove their name(s) from the amended Policy.
		\item At any point during the Policy Meeting, and with the permission of all other Amender(s) (or with nobody's permission if the Policy is has been left without a submitter), someone may add themselves to the Amender(s) list and make further amendments.
		\item If the Policy is without Amender(s), with nobody willing to become a member of Amender(s), discussion is dropped and the Policy is left as-is.
		\item At any time during discussion, the Endorser may cease endorsing the Policy amendment. If no Committee Members endorse the Policy amendment, discussion is dropped.
	\end{enumerate}
	\item Once all Submitter(s) and the Endorser are satisfied with the new state of the Policy, The new Policy will go to a vote by the Committee Members present at the Policy Meeting. If a Committee member is not present at the Policy Meeting, they forfeit their right to vote on said Policy amendment.
	\begin{enumerate}[i)]
		\item Any two (2) Committee Members attending the Policy Meeting may request to have voting postponed until the next Committee meeting, which becomes the Policy Meeting.
		\item If said Committee member(s) are not present at the next Policy Meeting, they forfeit their right to vote on the Policy.
		\item A Policy Meeting may not be postponed more than once per Policy amendment.
	\end{enumerate}
	\item The amended Policy is considered Enacted when it receives a Majority vote of approval by the Committee Members attending the Policy Meeting. The now-second-newest Amendment of the Policy will no longer be considered as Enacted.
\end{enumerate}

\subsection{Record of Policies}

The Secretary must keep a record of current, nullified, and rejected Policies, as well as Policies undergoing discussion by the Committee.

This record must include:

\begin{enumerate}[a)]
	\item An accurate and factual record of all Policies, as voted on by the Committee.
	\item The Submitter(s) and Endorser responsible for each Policy.
	\item All amendments (past and present) of any Policies that have been amended.
	\item An objective summary of the changes that took place with each amendment.
	\item The Amenders and Endorser responsible for each amendment.
\end{enumerate}

\subsection{Restriction of Policy}

All Policy is restricted by certain terms, and is to be held below the Constitution. Policy may not:

\begin{enumerate}[a)]
	\item Contradict or nullify anything set forth in the Constitution, or
	\item Contradict or nullify anything set forth in any rules or regulations the Constitution is restricted by. See the section ``Limit of the ComSSA Constitution''.
\end{enumerate}

\subsection{Mandatory Policies}

In order to ensure proper operation of the Club, certain Policies are required.

\subsubsection{Spending Policy}

The Spending Policy defines and regulates the use of funds by the Committee.

The Treasurer must appear as a Submitter on the initial Spending Policy, and an Amender on any amendments.

The Spending Policy must enact the following clauses:
\begin{enumerate}[a)]
	\item The assets and income of the Club shall be applied solely in furtherance of its above-mentioned objectives and no portion shall be distributed directly or indirectly to the Members of the Club except as bona fide compensation for services rendered or expenses incurred on behalf of the Club.
	\item In the event of the Club being dissolved, the amount that remains after such dissolution and the satisfaction of all debts and liabilities shall be transferred to the Curtin Student Guild.
\end{enumerate}

\section{The Constitution}

\subsection{Modification}

The Constitution may only be modified during an AGM or SGM upon the approval of no less than 70\% of Member votes.

\end{document}
