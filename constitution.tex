% Please adhere to having exactly one clause or sentence per line of source.
% At the expense of 80-column purity, this makes diff output much more useful.

\documentclass[a4paper,12pt]{article}
\usepackage[utf8]{inputenc}
\usepackage[T1]{fontenc}
\usepackage{parskip}
\usepackage{hyperref}
\usepackage{graphicx}
\usepackage{enumerate}

\hypersetup{
	pdfauthor=ComSSA,
	pdftitle=The Constitution of the Computer Science Students Association
}

\title{\scshape
	The Constitution\\
	of the\\
	Computer Science Students Association
}

\date{March 1, 2011}

\author{} % suppress 'LaTeX Warning: No \author given.'

\pagenumbering{gobble}
\thispdfpagelabel0

\raggedright

\begin{document}

\maketitle

\vspace{1in}

\begin{center}
	\includegraphics[width=3in]{logo/comssalogo_black_large.png}
\end{center}

\newpage

\pagenumbering{arabic}

\section{Preliminary}

\subsection{Name}

The club shall be known as the \textit{Computer Science Students Association}, also by the short name \textit{ComSSA} and hereafter referred to as ComSSA.

\subsection{Objective}

The objective of this constitution is to:

\begin{enumerate}[a)]
	\item Provide a framework for the operation of the club, and
	\item Assist the members who operate the club in a way that is consistent with the ethos of the club.
\end{enumerate}

\subsection{Definition and Interpretation}

\textbf{``ComSSA''} is the short name for the Computer Science Students Association.

\textbf{``Club''} refers to ComSSA.

\textbf{``AGM''} stands for ``Annual General Meeting''.

\textbf{``SGM''} stands for ``Special General Meeting''.

\textbf{``Club year''} refers to the period of time between the AGM of one year until the AGM of the next year.

\textbf{``OCM''} stands for ``Ordinary Committee Member''.

\textbf{``Member''} refers to any member of ComSSA.

\textbf{``Club asset''} refers to any item belonging to ComSSA, be it monetary, physical, abstract or otherwise.

\textbf{``Majority''} is defined as the nearest whole number to half of a cardinality, that is strictly greater than half of that cardinality.

\subsection{Registration}

As ComSSA is a part of Curtin University, it must officially register each year by being at least one of the following:

\begin{enumerate}[a)]
	\item A Curtin Student Guild registered club,
	\item A Curtin Student Guild affiliated club, and/or
	\item Registered with the Curtin Council.
\end{enumerate}

\subsection{Limit of the ComSSA Constitution}

This constitution is restricted and subject to the Curtin University Statutes and the Curtin Student Guild Regulations.

\section{Objectives of the Club}

\subsection{Objectives}

The objectives of the Club are:

\begin{enumerate}[a)]
	\item To provide social events for students and staff of the Department of Computing,
	\item To provide support for students entering university life,
	\item To provide support for students who need representation with the Department of Computing,
	\item To liaise with the Department of Computing and function as a forum for student and staff relations, and
	\item To assist in the provision of study and academically related material.
\end{enumerate}

\section{Membership}

\subsection{Limits of Membership}

Membership of the Club is unrestricted excepting a unanimous vote by the Committee.

\subsection{Suspension or Termination of Membership}

Excepting a unanimous vote by the Committee, membership of any person may not be suspended or terminated by any individual or group within the Club.

\subsection{Membership Length}

All memberships shall remain active from full payment of the membership fee until midnight of March 31st of the following year.

\section{Fees and Membership Costs}

\subsection{Membership Fee}

The membership fee for students with a Curtin Student Guild membership shall be five (5) Australian Dollars, or a Curtin Student Guild club registration voucher.

Registration for all other members shall be ten (10) Australian dollars.

\subsection{Compulsory Fees}

There shall be no compulsory fees charged to members apart from the membership fee.

\subsection{Costs for Members to Attend ComSSA Events}

Costs for members to attend any official ComSSA event shall be a maximum of five (5) Australian dollars or 20\% above cost price, whichever is greater.

\section{Committee}

\subsection{Committee Structure}

The Committee shall consist of the following voting office bearers:

\begin{enumerate}[a)]
	\item President,
	\item Vice-President,
	\item Secretary, and
	\item Treasurer.
\end{enumerate}

The Committee shall consist of the following voting non-office bearers:

\begin{enumerate}[a)]
	\setcounter{enumi}{4}
	\item Three (3) to five (5) Ordinary Committee Members (OCMs).
\end{enumerate}

\subsection{Length of Term}

Positions shall be held for one year; from the last day of the second semester (including the exam weeks) to the same time in the following year.

\subsection{Executive Members}

To be considered for the office bearer positions, a Nominee must meet all of the following criteria:

\begin{enumerate}[a)]
	\item The Nominee must be a current student at Curtin University, studying at the Bentley campus,
	\item The Nominee must be a Curtin Student Guild member, and
	\item At least one of the following must be true for the Nominee:
	\begin{enumerate}[i)]
		\item The Nominee must be studying at least one course which has greater than 50\% of its credit points assigned to units run by the Department of Computing, and/or
		\item The Nominee must be studying at least one course which will yield a degree which is deemed as \"relevant to computing\" by a Lecturer in the Department of Computing.
		\item The Nominee must be studying for a degree which is deemed as relevant to the interests of the club by way of a 70\% positive vote at an AGM.
	\end{enumerate}
\end{enumerate}

\subsection{Committee Eligibility}

All members of the Committee must be members of ComSSA. All members of the Committee must also agree to sign and abide by the annual Spending Policy (section 12).

\subsection{Ordinary Committee Members}

The method of selection for Ordinary Committee Member positions shall be determined by the Committee, and shall be limited to either:

\begin{enumerate}[a)]
	\item A majority vote at an AGM or SGM, or
	\item Appointment by the Committee.
\end{enumerate}

\subsection{Quorum}

Quorum shall be three (3) executives and two (2) other members of the Committee.

\subsection{Vacancy of Position}

Any position will become vacant upon any of the following criteria:

\begin{enumerate}[a)]
	\item The current occupant's written notice to the Committee,
	\item Missing of three (3) consecutive meetings without given notice, upon verification of the Committee or a delegate,
	\item Missing of six (6) consecutive meetings, upon verification of the Committee or a delegate, or
	\item No longer being a member of ComSSA.
\end{enumerate}

\subsection{Vacancy or Incapacitation}

\begin{enumerate}[a)]
	\item If any executive member vacates their position or becomes incapacitated, then the President shall acquire all of their powers and responsibilities until an SGM is called and the position is filled.
	\item The President may appoint an interim from the membership to temporarily fill the position until an SGM is called with the approval of the Committee.
	\item An SGM must be called within one (1) month of vacancy of a position.
\end{enumerate}

\section{Powers and Responsibilities}

\subsection{Duties of the Committee}

The Committee has the power and responsibility to:

\begin{enumerate}[a)]
	\item Meet regularly,
	\item Plan activities in accordance with the club objectives and inform all members,
	\item Formulate policy, rules and regulations in accordance with the club objectives,
	\item Raise and spend funds in accordance with the current spending policy,
	\item Accept or decline members who apply for an OCM position,
	\item Elect a member to take a vacant OCM position if necessary,
	\item Call for Special General Meetings,
	\item Act in the best interests of the club,
	\item Operate the club in an ethical manner, and
	\item Cancel the membership of any member upon any of the following:
	\begin{enumerate}[i)]
		\item Written agreement and support of the Curtin Student Guild President,
		\item Written agreement and support of the Curtin Student Guild Activities Vice-President, or
		\item A unanimous vote by the Committee.
	\end{enumerate}
\end{enumerate}

\subsection{Duties of the President}

The President has the power and responsibility to:

\begin{enumerate}[a)]
	\item Represent the club in any matters relating to ComSSA,
	\item Co-ordinate and supervise the Committee,
	\item Familiarise the members with the objectives, activities, rules and regulations of the club,
	\item Suspend an OCM from active duty, with the approval of one other executive member of the Committee,
	\item Suspend an executive member from active duty, with the approval of two other executive members of the Committee,
	\item Call for Special General Meetings,
	\item Provide the agenda at least 48 hours before a committee meeting, and
	\item Preside over committee meetings as the chairperson.
\end{enumerate}

\subsection{Duties of the Vice-President}

The Vice-President has the power and responsibility to:

\begin{enumerate}[a)]
	\item Assist the President in carrying out their responsibilities,
	\item Take on the President's duties upon the incapacitation of the President, or the vacancy of the President's position,
	\item Familiarise the members with the objectives, activities, rules and regulations of the club,
	\item Suspend an OCM from active duty, with the approval of the President,
	\item Suspend an executive member from active duty, with the approval of two other executive members of the Committee, and
	\item Preside over a committee meeting as its chairperson upon vacancy of the President.
\end{enumerate}

\subsection{Duties of the Secretary}

The Secretary has the power and responsibility to:

\begin{enumerate}[a)]
	\item Record and keep minutes of all meetings, and to ensure that those minutes are signed by the chairperson of the meeting,
	\item Manage correspondence of the Club,
	\item Manage a list of current members,
	\item Keep and provide to any member the Constitution, policies, rules and regulations of the club, and
	\item Suspend an OCM from active duty, with the approval of the President.
\end{enumerate}

\subsection{Duties of the Treasurer}

The Treasurer has the power and responsibility to:

\begin{enumerate}[a)]
	\item Keep a book of accounts dealing with property and finances of the club,
	\item Prepare a report of income and expenditure during the club year to be provided during an AGM or SGM,
	\item Ensure the Committee's adherence to the annual Spending Policy,
	\item Familiarise and report to the Committee the financial status of the club,
	\item Suspend an OCM from active duty, with the approval of the President, and
	\item Suspend a committee member upon a violation of the spending policy.
\end{enumerate}

\subsection{Duties of Ordinary Committee Members}

The OCMs have the power and responsibility to:

\begin{enumerate}[a)]
	\item Perform tasks as delegated by the Committee,
	\item Assist the Committee in their duties, and
	\item Ensure the proper conduct of the executive members.
\end{enumerate}

\subsection{Duties of the Members}

Members have the power and responsibility to:

\begin{enumerate}[a)]
	\item Vote during an AGM or SGM, and
	\item Become part of the Committee.
\end{enumerate}

\section{Removal and Suspension of Ordinary Committee Members}

\subsection{Suspension of Ordinary Committee Members}

\begin{enumerate}[a)]
	\item The President and one other executive member can suspend any OCM from active duty.
	\item Upon suspension of duty, the OCM must yield any club assets in their possession to the executive members.
\end{enumerate}

\subsection{Confirmation of Suspension}

A committee meeting must be held within fourteen (14) days of the suspension of an OCM.

\begin{enumerate}[a)]
	\item The President and executive responsible for suspending the OCM must present their reasons for suspending the OCM to the Committee.
	\item The OCM may explain their actions to the Committee.
	\item Suspension will be confirmed upon a majority vote by the full Committee, with the exception of the suspended OCM.
	\begin{enumerate}[i)]
		\item The Committee can request that the suspended OCM be removed from the meeting during the vote.
		\item The Committee can request that the suspended OCM be removed from the meeting during discussion.
	\end{enumerate}
\end{enumerate}

\subsection{Removal of Ordinary Committee Members}

Ordinary Committee Members can be removed from their position in any of the following ways:

\begin{enumerate}[a)]
	\item Confirmation of suspension for an OCM suspended from active duty,
	\item Refusal to sign or abide by the Spending Policy, or
	\item A majority vote by the full Committee, with the exception of the OCM being considered for removal:
	\begin{enumerate}[i)]
		\item The Committee can request that the OCM in question be removed from the meeting during the vote.
		\item The Committee can request that the OCM in question be removed from the meeting during discussion, but the OCM must be given a chance to explain their actions.
	\end{enumerate}
\end{enumerate}

Upon removal from the Committee:

\begin{enumerate}[a)]
	\setcounter{enumi}{3}
	\item The former OCM must yield any club assets in their possession to the executive members.
	\item The Committee must supply to the former OCM justification for removal.
\end{enumerate}

\subsection{Right to Appeal}

Any OCM removed from their position has the right to an appeal.

\begin{enumerate}[a)]
	\item The Committee must be notified of appeal within seven (7) days of removal.
	\item A committee meeting must be held within fourteen (14) days after receipt of appeal.
	\item The Committee cannot fill the vacant position for seven (7) days after removal, and any period until the meeting to consider appeal.
	\item The removed OCM must justify why their removal was unwarranted at the committee meeting.
	\item The OCM will be restored to their position upon a majority vote by the full Committee.
	\begin{enumerate}[i)]
		\item The Committee can request that the removed OCM be removed from the meeting during discussion.
		\item The Committee can request that the removed OCM be removed from the meeting during the vote.
	\end{enumerate}
	\item The Committee must supply to the former OCM justification for rejection of appeal.
\end{enumerate}

\section{Removal of Executive Members}

\subsection{Requests for Removal}

Any Member may request for the removal of an executive member from office. The request must be given in writing, and will only be considered if:

\begin{enumerate}[a)]
	\item The request includes reasons for removal from office, and
	\item The request includes the signatures of at least ten (10) members.
\end{enumerate}

\subsection{Consideration of Request}

A committee meeting must be held within fourteen (14) days of receipt of request.

\begin{enumerate}[a)]
	\item The Member(s) may present their reasons in person to the Committee.
	\item The executive member in question may explain their actions to the Committee and the Member(s) in attendance.
	\item The executive member will be removed from their position upon a majority vote by the full Committee, with the exception of the executive member being considered for removal.
	\begin{enumerate}[i)]
		\item The Committee can request that the executive member in question, and any Members be removed from the meeting during discussion.
		\item The Committee can request that the executive member in question, and any Members be removed from the meeting during the vote.
	\end{enumerate}
	\item The Committee must supply to the executive member in question, and the raising Member(s), justification as to the decision made.
\end{enumerate}

\subsection{Removal of Executive Members}

An executive member can be removed from office in any of the following ways:

\begin{enumerate}[a)]
	\item Granting of request for removal,
	\item Refusal to sign or abide by the Spending Policy, or
	\item A unanimous vote by the Committee with the exception of the executive member being considered for removal:
	\begin{enumerate}[i)]
		\item The Committee can request that the executive member in question be removed from the meeting during the vote.
		\item The Committee can request that the executive member in question be removed from the meeting during discussion, but the executive member must be given a chance to explain their actions.
	\end{enumerate}
\end{enumerate}

Upon removal from the Committee:

\begin{enumerate}[a)]
	\setcounter{enumi}{3}
	\item The former executive member must yield any club assets in their possession to the remaining executive members.
	\item The Committee must supply to the former executive member justification for removal.
	\item An SGM must be called within one (1) month of removal of position.
\end{enumerate}

\subsection{Right to Appeal}

\begin{enumerate}[a)]
	\item A removed executive member may stand for re-election at the SGM, and as such has no right to appeal.
	\item If the raising Member(s) are unsatisfied with the result of consideration, they may provide written notice of an SGM to the Committee (as specified in section 11).
	\begin{enumerate}[i)]
		\item The Committee must call an SGM within one (1) month upon receipt of notice, to discuss the matter of removal with the membership, and election of position if necessary.
	\end{enumerate}
\end{enumerate}

\section{Dismissal of Committee}

\subsection{Request for Dismissal}

Any Member may request for the dismissal of the entire Committee. The request must be given in writing, and will only be considered if:

\begin{enumerate}[a)]
	\item The request includes reasons for removal from office, and
	\item The request includes the signatures of at least twenty (20) members.
\end{enumerate}

\subsection{Receipt of Request}

An SGM must be called within fourteen (14) days of receipt of request. The agenda shall include:

\begin{enumerate}[a)]
	\item Justification for dismissal of the Committee by the Member(s) making the reqest,
	\item The Committee explanation for the actions detailed in the justification, and
	\item A vote for dismissal of the Committee.
	\begin{enumerate}[i)]
		\item The method of voting for dismissal of the Committee shall be a secret ballot supervised by an independent source.
	\end{enumerate}
	\item Election of new Committee (if necessary).
\end{enumerate}

\subsection{Confirmation of Dismissal}

Upon dismissal of the Committee:

\begin{enumerate}[a)]
	\item The dismissed Committee must yield any club assets in their possession to the newly elected Committee.
\end{enumerate}

\subsection{Right to Appeal}

The dismissed Committee may stand for re-election at the SGM, and as such has no right to appeal.

\section{Annual General Meetings}

\subsection{Time to be Held}

The AGM is to be held in the last three weeks of semester two (2).

\subsection{Notice to be Given}

At least fourteen (14) days of notice is to be given to all members detailing date, time and location of an AGM.

\subsection{Agenda}

The agenda shall consist of but is not limited to:

\begin{enumerate}[a)]
	\item A report from each executive member of the Committee,
	\item Ratification of the Constitution,
	\item Election of office bearers (executive members), and
	\item Any matter that is in writing, signed by at least ten (10) members.
\end{enumerate}

\subsection{Election of Positions}

The method of election shall be a secret ballot supervised by an independent source.

\subsection{Quorum}

Quorum for the AGM shall be twenty (20) members or a majority of the listed members, whichever is fewer.

\subsection{Voting}

\begin{enumerate}[a)]
	\item The method of voting, with the exception of election of positions, shall be decided by the President, or in their absence, the Vice-President.
	\item Any matter that is voted upon shall be deemed to be passed if a majority of attending votes are in favour.
\end{enumerate}

\section{Special General Meetings}

\subsection{Alternative Names}

An SGM may also be known as:

\begin{enumerate}[a)]
	\item Extraordinary General Meeting (EGM), or
	\item Emergency General Meeting (EGM).
\end{enumerate}

\subsection{Convening}

An SGM can be called for by any of the following means:

\begin{enumerate}[a)]
	\item Written notice by a majority of the Committee,
	\item Written notice from the President, or
	\item Written notice by at least ten (10) members of ComSSA.
\end{enumerate}

\subsection{Notice to be Given}

At least seven (7) days of notice is to be given to all members detailing date, time and location of an SGM.

\subsection{Agenda}

The agenda shall consist of:

\begin{enumerate}[a)]
	\item A report from each executive member,
	\item Election of any vacant office bearer positions,
	\item Any matter in writing signed by at least ten (10) members, and
	\item Any matter in writing signed by a majority of the Committee.
\end{enumerate}

\subsection{Quorum}

Quorum for an SGM shall be twenty (20) members or a majority of the listed members, whichever is fewer.

\subsection{Election of Positions}

The method of election shall be a secret ballot supervised by an independent source.

\subsection{Voting}

\begin{enumerate}[a)]
	\item The method of voting, with the exception of election of positions and dismissal of the Committee, shall be decided by the President, or in their absence, the Vice-President.
	\item Any matter that is voted upon shall be deemed to be passed if a majority of attending votes are in favour.
\end{enumerate}

\section{Spending Policy}

\subsection{Drafting of Spending Policy}

\begin{enumerate}[a)]
	\item The Spending Policy will be drafted annually within one (1) month of the AGM by the new Treasurer.
	\item The Spending Policy defines and regulates the use of funds by the Committee, and will last the duration of a club year.
\end{enumerate}

\subsection{Acceptance of Spending Policy}

The Spending Policy will be confirmed upon receipt of the signatures of all members of the Committee, both executive and ordinary. Any member who refuses to sign the Spending Policy will be removed after a majority vote by the Committee.

\subsection{Violation of Spending Policy}

Adherence to the Spending Policy will be regulated by the Treasurer.

Violation of the spending policy may result in suspension and possible removal from the Committee.

\subsection{Non-Profit Clause}

The assets and income of the Club shall be applied solely in furtherance of its above-mentioned objectives and no portion shall be distributed directly or indirectly to the members of the Club except as bona fide compensation for services rendered or expenses incurred on behalf of the Club.

\subsection{Dissolution Clause}

In the event of the Club being dissolved, the amount that remains after such dissolution and the satisfaction of all debts and liabilities shall be transferred to another organisation with similar purposes which is not carried on for the profit or gain of its individual members.

\section{The Constitution}

\subsection{Modification}

The Constitution may only be modified during an AGM or SGM upon the approval of no less than 70\% of votes.

\end{document}
